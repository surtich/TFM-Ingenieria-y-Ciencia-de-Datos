\cleardoublepage
\selectlanguage{spanish}

\newglossaryentry{escala de likert}
{
    name=escala de Likert,
    description={Es una escala de evaluación ordinal utilizada en cuestionarios. Suele constar de 5 a 7 niveles desde "Totalmente en desacuerdo" hasta "Totalmente de acuerdo."}
}

\newglossaryentry{accesibilidad}
{
    name=accesibilidad,
    description={Condición que deben cumplir los entornos, productos y servicios para que sean comprensibles, utilizables y practicables por todos los ciudadanos, incluidas las personas con discapacidad.}
}


\newglossaryentry{mooc}
{
    name=MOOC,
    description={Curso en Línea Masivo y Abierto}
}


\newglossaryentry{sha}
{
    name=SHA,
    description={Secure Hash Algorithm.
    Es un algoritmo criptográfico cuyo propósito es generar un resumen único (hash) de mensaje
    que garantiza la integridad de los datos.
    }
}


\newglossaryentry{crc}
{
    name=CRC,
    description={Comprobación de Redundancia Cíclica.
    Es un algoritmo de detección de errores utilizado para garantizar la integridad de la información.
    }
}


\newglossaryentry{subtitulado cerrado}
{
    name=subtitulado cerrado,
    description={Subtitulado que incluye descripciones de sonidos y otros elementos relevantes en un contenido audiovisual.
    El subtitulado cerrado es útil para personas con discapacidad auditiva o para aquellos que deseen ver
    el contenido en lugares donde no se puede reproducir el sonido, como entornos ruidosos o silenciosos.
    }
}


\newglossaryentry{triple ciego}
{
    name=triple ciego,
    description={Diseño de experimento en el que tanto los participantes,
    como los investigadores encargados de administrar el tratamiento y
    los evaluadores de los resultados, desconocen quién está recibiendo cada nivel de tratamiento.}
}


\newglossaryentry{odds}
{
    name=odds,
    description={
        Es la proporción entre dos probabilidades complementarias. Matemáticamente:
    $$
        odds = \frac{P(evento)}{1- P(evento)}
    $$
    El odds puede tomar valores en el rango cero a infinito.
    Si el valor del odds es 1, significa que la probabilidad del evento y su complementario son iguales.
    Si el odds es mayor que 1, indica que la probabilidad del evento es mayor que la probabilidad de su complementario.
    Por el contrario, si el odds es menor que 1, significa que la probabilidad del evento es menor que
    la probabilidad de su complementario.}
}

\newglossaryentry{odds ratio}
{
    name=odds ratio,
    description={
        Es una medida de asociación entre dos variables dicotómicas.
        Matemáticamente se define como la razón de dos \gls{odds}:
        $$
        \begin{aligned}
        odds_{Y=1} &= \frac{P(X=1|Y=1)}{1-P(X=1|Y=1)}\\
        odds_{Y=0} &= \frac{P(X=1|Y=0)}{1-P(X=1|Y=0)}\\
        OR = OR_Y = OR_X &=\frac{\frac{P(X=1|Y=1)}{1-P(X=1|Y=1)}}{\frac{P(X=1|Y=0)}{1-P(X=1|Y=0)}}
        \end{aligned}
        $$
        Si el OR es igual a 1, indica que no hay asociación entre las variables.
        Si el OR es mayor que 1, indica una mayor probabilidad del $odds_{Y=1}$.
        Si el OR es mayor que 1, indica una mayor probabilidad del $odds_{Y=0}$.
    }
}


\newglossaryentry{OR}
{
    name=OR,
    description={odds ratio},
    see={odds ratio}
}


\newglossaryentry{GLM}
{
    name=GLM,
    description={
        Modelo Lineal Generalizado.
        Es una extensión del Modelo Lineal General que permite variables de respuesta que siguen distribuciones
        de probabilidad no normales.
    }
}


\newglossaryentry{R}
{
    name=R,
    description={Lenguaje de programación especializado en análisis estadístico y la generación de gráficos.}
}



\newglossaryentry{diseño completamente aleatorizado}
{
    name=diseño completamente aleatorizado,
    description={
        Diseño experimental en el cual los sujetos de estudio o las unidades experimentales se asignan de manera aleatoria
        a diferentes tratamientos o grupos de tratamiento.
        En este tipo de diseño, cada unidad experimental tiene la misma probabilidad de ser asignada a cualquier
        tratamiento en particular. Se utiliza cuando no hay restricciones o consideraciones específicas sobre cómo
        se deben asignar las unidades experimentales a los tratamientos.
        Se asume que las unidades experimentales son homogéneas y que no hay factores de confusión o
        variables ocultas que puedan influir en los resultados del experimento.
        Este diseño es útil cuando el objetivo principal del estudio es comparar los efectos de diferentes
        tratamientos. Al asignar las unidades de forma aleatoria, se busca reducir el sesgo y controlar
        los factores de confusión desconocidos, ya que se espera que las diferencias observadas entre los grupos
        sean principalmente el resultado de los tratamientos aplicados. El diseño completamente aleatorizado
        es considerado uno de los diseños experimentales más simples y robustos.
        Sin embargo, puede requerir un tamaño de muestra más grande para detectar diferencias significativas
        entre los grupos de tratamiento debido a la aleatorización pura y la posible variabilidad
        inherente en los datos.
        }
}


\newglossaryentry{diseño cruzado}
{
    name=diseño cruzado,
    description={
        Diseño experimental en el cual los sujetos reciben diferentes tratamientos en distintas varios momentos o periodos.
        Se realiza una asignación aleatoria para determinar el orden en el cual los tratamientos son recibidos.
        El diseño más simple de este tipo involucra dos grupos de sujetos,
        uno de los cuales recibe cada uno de dos tratamientos, A y B, en el orden AB,
        mientras que el otro los recibe en orden inverso, BA.
        Dado que la comparación de tratamientos se realiza "dentro del sujeto" en lugar de "entre sujetos", es probable
        que se necesiten menos sujetos para lograr un poder estadístico determinado.
        El análisis de estos diseños no es necesariamente sencillo debido a la posibilidad de efectos de secuencia y periodo.
        Para minimizar este problema, a menudo se deja un período de lavado entre los periodos.
    }
}


\newglossaryentry{efecto secuencia}
{
    name=efecto secuencia,
    description={
        En un \gls{diseño cruzado} es el efecto que se produce cuando el orden de las intervenciones afecta el resultado final.
    }
}


\newglossaryentry{efecto periodo}
{
    name=efecto periodo,
    description={
        En un diseño experimental es el efecto que se produce cuando la aplicación de un tratamiento en un periodo
        es afectado por la aplicación de otro tratamiento en un periodo anterior.
    }
}

\newglossaryentry{ANOVA}
{
    name=ANOVA,
    description={
        El Análisis de la Varianza es un método estadístico para comparar las medias de dos o más grupos
        comparando la varianza de los datos entre grupos e intra grupos.
    }
}


\newglossaryentry{MANOVA}
{
    name=MANOVA,
    description={
        El Análisis de Varianza Multivariado una extensión de \gls{ANOVA}
        que se utiliza cuando se tienen múltiples variables respuesta.
    }
}



\newglossaryentry{error de tipo I}
{
    name=error de tipo I,
    description={
        El error de tipo I o falso positivo es el error que se comete cuando se rechaza la hipótesis nula
        siendo esta verdadera en la población.
    }
}


\newglossaryentry{error de tipo II}
{
    name=error de tipo II,
    description={
        El error de tipo II o falso negativo es el error que se comete cuando no se rechaza la hipótesis nula
        siendo esta falsa en la población.
    }
}


\newglossaryentry{MLE}
{
    name=MLE,
    description={
        La Estimación de Máxima Verosimilitud es un método estadístico de estimación paramétrica
        basado en la maximización de la función de verosimilitud de los datos observados.
        La verosimilitud es la probabilidad de observar los datos en función de los parámetros del modelo.
        El objetivo del MLE es encontrar los valores de los parámetros que maximizan
        la probabilidad de obtener los datos observados.
        En otras palabras, se busca encontrar los valores de los parámetros que hacen que los datos observados
        sean más probables bajo el modelo propuesto.
        En muchos casos, es más conveniente maximizar el logaritmo de la función de verosimilitud,
        ya que simplifica los cálculos y no afecta la ubicación de los máximos por ser la función logarítmica
        monótona creciente.        
        }
}


\newglossaryentry{Regresión Logística}
{
    name=Regresión Logística,
    description={
        Modelo estadístico en el que la variable respuesta es binaria o dicotómica. Es un tipo particular de \gls{GLM}.}
}


\newglossaryentry{Regresión Ordinal}
{
    name=Regresión Ordinal,
    description={
        Modelo estadístico en el que la variable respuesta es ordinal. Es una extensión de la \gls{Regresión Logística}.}
}


\newglossaryentry{función logit}
{
    name=función logit,
    description={
        Función matemática utilizada en el contexto de la \gls{Regresión Logística} para transformar
        la probabilidad de un evento en un valor continuo que abarca todo el intervalo real.
        La función logit se define como el logaritmo natural del odds de un evento:        
        $$
        logit(evento) = log (\frac{P(evento)}{1-P(evento)})
        $$
    }
}


\newglossaryentry{función logística}
{
    name=función logística,
    description={
        También conocida como función sigmoide, es la función inversa de la \gls{función logit}.
        es una función matemática que transforma una variable continua en un rango de valores entre 0 y 1.
        Esta función se utiliza comúnmente en la \gls{Regresión Logística} para convertir los resultados de una combinación lineal
        de variables predictoras en probabilidades:
        $$
        P(x) = \frac{1}{1 + e^{-x}}
        $$
    }
}



\newglossaryentry{LRT}
{
    name=LRT,
    description={
        El Test de Razón de Verosimilitudes en español, es una prueba estadística utilizada
        para comparar dos modelos estadísticos y determinar cuál de ellos proporciona un mejor ajuste
        a los datos observados.
        La idea detrás del LRT es comparar la verosimilitud del modelo completo con la verosimilitud
        del modelo reducido, que es un modelo en el que algunos de los parámetros se han igualado a cero.
        Si la diferencia entre estas verosimilitudes es estadísticamente significativa,
        se concluye que el modelo completo proporciona un mejor ajuste a los datos que el modelo reducido.
    }
}


\newglossaryentry{GLMM}
{
    name=GLMM,
    description={
        Modelo estadístico también conocido como Modelo Multinivel o Modelo Jerárquico,
        es una extensión de los modelos \gls{GLM} utilizado para analizar datos estructurados en diferentes niveles o niveles jerárquicos.
        Este tipo de modelo es útil cuando los datos tienen una estructura anidada, como estudiantes dentro de escuelas,
        pacientes dentro de hospitales o empleados dentro de empresas.
        En un modelo multinivel, se reconoce que los datos están agrupados en diferentes niveles
        y que las observaciones dentro de cada nivel son dependientes debido a la influencia del nivel
        al que pertenecen. Por ejemplo, las puntuaciones de los estudiantes dentro de una escuela pueden no ser independientes
        debido a la influencia del entorno escolar.
        El modelo multinivel permite modelar tanto las variaciones entre los diferentes niveles
        (variaciones entre escuelas, hospitales, empresas, etc.)
        como las variaciones dentro de cada nivel (variaciones individuales dentro de cada escuela, hospital, empresa, etc.).
        Esto se logra mediante la inclusión de términos aleatorios en el modelo que capturan
        las variaciones entre los niveles. La ventaja de utilizar un modelo multinivel es que tiene en cuenta la estructura jerárquica de los datos y permite estimar los efectos tanto a nivel individual como a nivel de grupo. La estimación de un modelo multinivel generalmente se realiza mediante métodos de Máxima Verosimilitud Restringida (REML) o mediante el enfoque de Estimación de Máxima Verosimilitud (MLE).
    }
}


\newglossaryentry{ICC}
{
    name=ICC,
    description={
        La Correlación Intraclase es una medida estadística utilizada para cuantificar la proporción de varianza total
        de una variable que se debe a la variación entre grupos. Se utiliza en el contexto de modelos \gls{GLMM}
        cuando las observaciones están agrupadas en diferentes unidades,
        y mide la parte de la varianza explicada por los
        efectos aleatorios.
    }
}

\newglossaryentry{MCMC}
{
    name=MCMC,
    description={
        Los Métodos de Montecarlo basados en cadenas de Markov es un método estadístico utilizado para simular
        muestras de una distribución de probabilidad.
        El método de MCMC se basa en la construcción de una cadena de Markov,
        que es una secuencia de valores generados a partir de una distribución de probabilidad condicional dada
        la observación anterior en la cadena.
        Estos valores se generan iterativamente, de manera que la distribución de probabilidad de los valores
        de la cadena converge a la distribución de interés.        
        La principal ventaja de MCMC es que permite aproximar
        y obtener muestras representativas de la distribución de probabilidad de interés,
        incluso cuando dicha distribución no se puede obtener de manera analítica o es
        computacionalmente costosa de calcular.
    }
}


\newglossaryentry{shrinkage}
{
    name=shrinkage,
    description={
        En el contexto de \gls{GLMM}, el término "shrinkage" (o "encogimiento" en español)
        se refiere a un proceso mediante el cual los efectos estimados a nivel individual se ajustan hacia el valor promedio del grupo al que pertenecen.}
}

\printglossaries