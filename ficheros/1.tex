% Options for packages loaded elsewhere
\PassOptionsToPackage{unicode}{hyperref}
\PassOptionsToPackage{hyphens}{url}
\PassOptionsToPackage{dvipsnames,svgnames,x11names}{xcolor}
%
\documentclass[
  letterpaper,
  DIV=11,
  numbers=noendperiod]{scrartcl}

\usepackage{amsmath,amssymb}
\usepackage{iftex}
\ifPDFTeX
  \usepackage[T1]{fontenc}
  \usepackage[utf8]{inputenc}
  \usepackage{textcomp} % provide euro and other symbols
\else % if luatex or xetex
  \usepackage{unicode-math}
  \defaultfontfeatures{Scale=MatchLowercase}
  \defaultfontfeatures[\rmfamily]{Ligatures=TeX,Scale=1}
\fi
\usepackage{lmodern}
\ifPDFTeX\else  
    % xetex/luatex font selection
\fi
% Use upquote if available, for straight quotes in verbatim environments
\IfFileExists{upquote.sty}{\usepackage{upquote}}{}
\IfFileExists{microtype.sty}{% use microtype if available
  \usepackage[]{microtype}
  \UseMicrotypeSet[protrusion]{basicmath} % disable protrusion for tt fonts
}{}
\makeatletter
\@ifundefined{KOMAClassName}{% if non-KOMA class
  \IfFileExists{parskip.sty}{%
    \usepackage{parskip}
  }{% else
    \setlength{\parindent}{0pt}
    \setlength{\parskip}{6pt plus 2pt minus 1pt}}
}{% if KOMA class
  \KOMAoptions{parskip=half}}
\makeatother
\usepackage{xcolor}
\setlength{\emergencystretch}{3em} % prevent overfull lines
\setcounter{secnumdepth}{-\maxdimen} % remove section numbering
% Make \paragraph and \subparagraph free-standing
\ifx\paragraph\undefined\else
  \let\oldparagraph\paragraph
  \renewcommand{\paragraph}[1]{\oldparagraph{#1}\mbox{}}
\fi
\ifx\subparagraph\undefined\else
  \let\oldsubparagraph\subparagraph
  \renewcommand{\subparagraph}[1]{\oldsubparagraph{#1}\mbox{}}
\fi


\providecommand{\tightlist}{%
  \setlength{\itemsep}{0pt}\setlength{\parskip}{0pt}}\usepackage{longtable,booktabs,array}
\usepackage{calc} % for calculating minipage widths
% Correct order of tables after \paragraph or \subparagraph
\usepackage{etoolbox}
\makeatletter
\patchcmd\longtable{\par}{\if@noskipsec\mbox{}\fi\par}{}{}
\makeatother
% Allow footnotes in longtable head/foot
\IfFileExists{footnotehyper.sty}{\usepackage{footnotehyper}}{\usepackage{footnote}}
\makesavenoteenv{longtable}
\usepackage{graphicx}
\makeatletter
\def\maxwidth{\ifdim\Gin@nat@width>\linewidth\linewidth\else\Gin@nat@width\fi}
\def\maxheight{\ifdim\Gin@nat@height>\textheight\textheight\else\Gin@nat@height\fi}
\makeatother
% Scale images if necessary, so that they will not overflow the page
% margins by default, and it is still possible to overwrite the defaults
% using explicit options in \includegraphics[width, height, ...]{}
\setkeys{Gin}{width=\maxwidth,height=\maxheight,keepaspectratio}
% Set default figure placement to htbp
\makeatletter
\def\fps@figure{htbp}
\makeatother

\KOMAoption{captions}{tableheading}
\makeatletter
\@ifpackageloaded{tcolorbox}{}{\usepackage[skins,breakable]{tcolorbox}}
\@ifpackageloaded{fontawesome5}{}{\usepackage{fontawesome5}}
\definecolor{quarto-callout-color}{HTML}{909090}
\definecolor{quarto-callout-note-color}{HTML}{0758E5}
\definecolor{quarto-callout-important-color}{HTML}{CC1914}
\definecolor{quarto-callout-warning-color}{HTML}{EB9113}
\definecolor{quarto-callout-tip-color}{HTML}{00A047}
\definecolor{quarto-callout-caution-color}{HTML}{FC5300}
\definecolor{quarto-callout-color-frame}{HTML}{acacac}
\definecolor{quarto-callout-note-color-frame}{HTML}{4582ec}
\definecolor{quarto-callout-important-color-frame}{HTML}{d9534f}
\definecolor{quarto-callout-warning-color-frame}{HTML}{f0ad4e}
\definecolor{quarto-callout-tip-color-frame}{HTML}{02b875}
\definecolor{quarto-callout-caution-color-frame}{HTML}{fd7e14}
\makeatother
\makeatletter
\makeatother
\makeatletter
\makeatother
\makeatletter
\@ifpackageloaded{caption}{}{\usepackage{caption}}
\AtBeginDocument{%
\ifdefined\contentsname
  \renewcommand*\contentsname{Table of contents}
\else
  \newcommand\contentsname{Table of contents}
\fi
\ifdefined\listfigurename
  \renewcommand*\listfigurename{List of Figures}
\else
  \newcommand\listfigurename{List of Figures}
\fi
\ifdefined\listtablename
  \renewcommand*\listtablename{List of Tables}
\else
  \newcommand\listtablename{List of Tables}
\fi
\ifdefined\figurename
  \renewcommand*\figurename{Figure}
\else
  \newcommand\figurename{Figure}
\fi
\ifdefined\tablename
  \renewcommand*\tablename{Table}
\else
  \newcommand\tablename{Table}
\fi
}
\@ifpackageloaded{float}{}{\usepackage{float}}
\floatstyle{ruled}
\@ifundefined{c@chapter}{\newfloat{codelisting}{h}{lop}}{\newfloat{codelisting}{h}{lop}[chapter]}
\floatname{codelisting}{Listing}
\newcommand*\listoflistings{\listof{codelisting}{List of Listings}}
\makeatother
\makeatletter
\@ifpackageloaded{caption}{}{\usepackage{caption}}
\@ifpackageloaded{subcaption}{}{\usepackage{subcaption}}
\makeatother
\makeatletter
\@ifpackageloaded{tcolorbox}{}{\usepackage[skins,breakable]{tcolorbox}}
\makeatother
\makeatletter
\@ifundefined{shadecolor}{\definecolor{shadecolor}{rgb}{.97, .97, .97}}
\makeatother
\makeatletter
\makeatother
\makeatletter
\makeatother
\ifLuaTeX
  \usepackage{selnolig}  % disable illegal ligatures
\fi
\IfFileExists{bookmark.sty}{\usepackage{bookmark}}{\usepackage{hyperref}}
\IfFileExists{xurl.sty}{\usepackage{xurl}}{} % add URL line breaks if available
\urlstyle{same} % disable monospaced font for URLs
\hypersetup{
  colorlinks=true,
  linkcolor={blue},
  filecolor={Maroon},
  citecolor={Blue},
  urlcolor={Blue},
  pdfcreator={LaTeX via pandoc}}

\author{}
\date{}

\begin{document}
\ifdefined\Shaded\renewenvironment{Shaded}{\begin{tcolorbox}[breakable, interior hidden, borderline west={3pt}{0pt}{shadecolor}, frame hidden, boxrule=0pt, sharp corners, enhanced]}{\end{tcolorbox}}\fi

\hypertarget{motivaciuxf3n}{%
\subsection{Motivación}\label{motivaciuxf3n}}

Algunas personas tienen problemas de \gls{accesibilidad} a contenidos
multimedia. Añadir subtítulos a los vídeos facilita que muchas de ellas
superen estas dificultades ya que permiten la percepción visual de
información que originalmente era acústica. Por ello, los subtítulos
ayudan a las personas sordas o con discapacidad auditiva a acceder al
contenido audiovisual y constituyen uno de los componentes fundamentales
de la accesibilidad audiovisual {[}ver @jperez1{]}. En las \emph{Pautas
de Accesibilidad para el Contenido Web, WCAG2.1} {[}ver @WCAG21{]} se
incluyen pautas y criterios que deben seguirse en el subtitulado. La
norma UNE 153010 {[}ver @aenor2012{]} sobre \emph{Subtitulado para
personas sordas y personas con discapacidad auditiva} ``especifica
requisitos y recomendaciones sobre la presentación del subtitulado para
personas sordas y personas con discapacidad auditiva como medio de apoyo
a la comunicación para facilitar la accesibilidad de los contenidos
audiovisuales de la Sociedad de la Información.''

Las plataformas de compartición de vídeos como Youtube o las de creación
de cursos \gls{mooc} como Coursera o EDX permiten en la actualidad la
generación de subtítulos automáticos. Además, existen comunidades que se
dedican a subtitular todo tipo de material multimedia. No obstante,
frecuentemente los subtítulos producidos de esta manera no tienen en
cuenta la necesidades de accesibilidad específicas de personas con
problemas de audición. Estas personas normalmente van a necesitar un
\gls{subtitulado cerrado} (\emph{closed caption}). Este subtitulado
consta de información interpretativa e incluye descripción textual de
efectos sonoros junto con otros elementos como la perfecta
sincronización con el hablante, el número de caracteres por línea, la
exactitud del diálogo, la ubicación, el tamaño o el contraste del
subtítulo. Así, @parton2016 realizó un estudio para determinar si los
subtítulos automáticos generados por YouTube cumplen las necesidades de
los estudiantes universitarios sordos. En el estudio se contabilizaron
un total de 525 errores en 68 minutos de video (una tasa de 7.7 errores
por minuto). La accesibilidad multimedia es, por lo tanto, una actividad
importante y compleja que requiere dedicación y conocimiento específico
y va más allá de verificar si el vídeo tiene o no subtítulos.

\hypertarget{sec-objetivos}{%
\subsection{Objetivos}\label{sec-objetivos}}

El objetivo de este trabajo es responder a la siguiente pregunta de
investigación:

\begin{tcolorbox}[enhanced jigsaw, titlerule=0mm, toptitle=1mm, bottomrule=.15mm, bottomtitle=1mm, breakable, opacityback=0, colback=white, rightrule=.15mm, leftrule=.75mm, coltitle=black, opacitybacktitle=0.6, colbacktitle=quarto-callout-note-color!10!white, title=\textcolor{quarto-callout-note-color}{\faInfo}\hspace{0.5em}{Pregunta de investigación}, colframe=quarto-callout-note-color-frame, arc=.35mm, toprule=.15mm, left=2mm]

¿Son los estudiantes de un curso de creación de materiales accesibles
capaces de encontrar diferencias en la calidad del subtitulado de un
vídeo?

\end{tcolorbox}

Los datos utilizados proceden de una actividad voluntaria sobre
evaluación del subtitulado que se propuso a los estudiantes de la Sexta
Edición de 2022 del \textbf{curso MOOC Materiales Digitales Accesibles}
perteneciente al Canal Fundación ONCE en UNED Abierta. Este curso es una
iniciativa del Real Patronato sobre Discapacidad en colaboración con la
Fundación ONCE y la UNED. El curso está dirigido por los profesores
Emilio Letón Molina y Alejandro Rodríguez Ascaso. El Canal Fundación
ONCE se creó en 2016 y hasta marzo de 2023 casi de 23.000 estudiantes se
han inscrito en alguno de sus cursos. En ellos se ofrece formación en
conocimientos y habilidades necesarios para diseñar productos, entornos,
sistemas y servicios desde una perspectiva de diseño universal:

\begin{itemize}
\tightlist
\item
  Reconocer las necesidades relacionadas con la accesibilidad.
\item
  Dar respuesta a estas necesidades, cada actor en la medida de sus
  posibilidades.
\item
  Integrar soluciones de accesibilidad universales y específicas
  utilizando la tecnología apropiada.
\end{itemize}

La Fundación ONCE, el Real Patronato sobre Discapacidad y la UNED
colaboran para producir recursos educativos abiertos, gratuitos, de
calidad y accesibles para todas las personas. Los cursos del Canal se
ofrecen a través de UNED Abierta, portal de formación en formato
\gls{mooc} sobre una instancia personalizada de la plataforma OpenEdX.
En la actualidad la oferta es de nueve cursos. En el año 2022 se
matricularon 3.764 alumnos y aprobaron el 19,34\% de los matriculados.

El curso de Materiales Digitales Accesibles se lleva realizando desde
2017, se han matriculado en alguna de sus siete ediciones (hasta mayo de
2023) más de 8.000 alumnos y tiene un porcentaje medio de aprobados
sobre matriculados del 14,6\%. En la Sexta Edición se matricularon 1.261
alumnos y aprobaron 165 (13,08\%).

@jperez2 realizaron una experiencia similar a la planteada en este
trabajo y llegaron a la conclusión de que evaluadores novatos pueden
identificar problemas de accesibilidad en vídeos. Por lo tanto, la
evaluación social propuesta por @kawanaka2008 se podría aplicar a los
contenidos de vídeo en la Web. Sin embargo, también constataron que los
evaluadores novatos pueden pasar por alto problemas sutiles de
accesibilidad que requieren un conocimiento experto como, por ejemplo,
la evaluación del contraste, ya que requieren herramientas adecuadas.
Por ello, en este trabajo también se responderá a los siguientes
objetivos específicos:

\begin{tcolorbox}[enhanced jigsaw, titlerule=0mm, toptitle=1mm, bottomrule=.15mm, bottomtitle=1mm, breakable, opacityback=0, colback=white, rightrule=.15mm, leftrule=.75mm, coltitle=black, opacitybacktitle=0.6, colbacktitle=quarto-callout-tip-color!10!white, title=\textcolor{quarto-callout-tip-color}{\faLightbulb}\hspace{0.5em}{Objetivos específicos}, colframe=quarto-callout-tip-color-frame, arc=.35mm, toprule=.15mm, left=2mm]

¿Qué aspectos de la calidad del subtitulado son más fácilmente
reconocibles por los estudiantes y en cuáles existe mayor dificultad?

\end{tcolorbox}

\hypertarget{descripciuxf3n-de-la-experiencia}{%
\subsection{Descripción de la
experiencia}\label{descripciuxf3n-de-la-experiencia}}

La actividad de subtitulado fue voluntaria y sin influencia en la
calificación final del alumno. Se realizó en el módulo ``Accesibilidad
del material multimedia''. En este mismo módulo, y antes de la actividad
de subtitulado, los alumnos hubieron completado las secciones
``Accesibilidad de la información sonora'' y ``Accesibilidad de la
información visual'', con lo cual adquirieron un conocimiento previo
sobre creación de vídeos y subtítulos accesibles. Se estima que los
estudiantes hubieron empleado unas tres horas de formación en
accesibilidad multimedia (una de ellas específicamente en subtitulado)
antes de realizar la actividad.

De acuerdo al \textbf{compromiso ético} del Canal los datos de los
estudiantes se han suministrado anonimizados usando un identificador
generado con SHA512. Además, se han eliminado del estudio los datos de
estudiantes que, a pesar de haber realizado la actividad de subtitulado,
no dieron su consentimiento para que sus datos fueran utilizados en
estudios científicos.

La actividad consistió en ver dos vídeos idénticos de 43 segundos que
solo se diferencian en la calidad del subtitulado. Los subtítulos del
vídeo bien subtitulado se realizaron por un experto de FIAPAS
(Confederación Española de Familias de Personas Sordas) siguiendo la
norma UNE 153010 {[}ver @aenor2012{]}. El otro vídeo tenía un
subtitulado similar pero se introdujeron pequeñas deficiencias, algunas
de ellas inapreciables para alguien que carezca de conocimientos sobre
accesibilidad. El orden de los vídeos fue aleatorio, de tal forma que
una cohorte de alumnos vio primero el vídeo bien subtitulado y luego el
mal subtitulado y la otra lo hizo al revés. Después de ver cada uno de
los vídeos, los alumnos respondieron a una escala de Likert de 5 niveles
y 18 ítems. Los 18 ítems de Likert responden a los criterios de la norma
UNE 153010 {[}ver @aenor2012{]}.

Los términos \textbf{escala de Likert} e ítem de Likert se prestan a
menudo a confusión ya que se utilizan con distintos significados. En
este trabajo se seguirá la convención más habitual {[}ver
@uebersax2006{]} de denominar ítem de Likert a cada una de las preguntas
de que consta un cuestionario o test, siendo la escala de Likert el
conjunto de todos los ítems del cuestionario. Cada ítem se contestó
marcando una opción de entre un conjunto ordenado de respuestas o
niveles propuesto e idéntico para todos los ítems. Por ello, se debe
evitar denominar escala a los niveles de un ítem.

El diseño del experimento fue \textbf{triple ciego}. Es decir, a los
alumnos no se les informó de si estaban viendo el vídeo con mejor o con
peor calidad de subtitulado. Los directores del MOOC tampoco conocieron
esta información, como tampoco se conocía en el momento de analizar los
datos, ya que los vídeos tienen identificaciones ofuscadas con CRC32b y
no contienen ninguna indicación del tipo de subtitulado del
vídeo\footnote{En la respuesta a cada ítem, el alumno pudo añadir
  comentarios. Éstos fueron eliminados en la fase de análisis para que
  no filtren información referente al tipo de subtitulado que el alumno
  creyó estar contestando y solo se utilizaron en la fase de discusión
  ver \textbf{?@sec-discusion}().}. El ``ciego fue liberado'' en la fase
de elaboración de la discusión de este trabajo (ver
\textbf{?@sec-discusion}).

\hypertarget{organizaciuxf3n-del-trabajo}{%
\subsection{Organización del
trabajo}\label{organizaciuxf3n-del-trabajo}}

En el capítulo \textbf{Marco teórico y estado del arte} (ver
\textbf{?@sec-arte}) se enmarca la actividad de subtitulado en el
contexto del modelado estadístico, describiendo sus principales
características y proponiendo y justificando las técnicas y modelos que
se van a utilizar. En este capítulo también se explica la forma en que
se deben interpretar y evaluar los modelos.

El capítulo \textbf{Materiales y métodos} (ver \textbf{?@sec-metodo})
describe los ficheros de datos suministrados, la actividad de
preprocesado realizada sobre los mismos y las variables que se
utilizarán en el modelado estadístico.

El capítulo \textbf{Modelado estadístico} (ver \textbf{?@sec-modelado})
comienza con un Análisis Exploratorio de los datos, tras el que se
describe cómo se han aplicado las técnicas de modelado presentadas en el
Marco Teórico al diseño del experimento de la actividad de subtitulado.

En el capítulo de \textbf{Resultados} (ver \textbf{?@sec-resultados}) se
presentan los resultados de los modelos seleccionados en el capítulo
anterior.

En el capítulo \textbf{Discusión} (ver \textbf{?@sec-discusion}) se
utilizan los resultados del capítulo anterior para responder a la
pregunta de investigación y a los objetivos específicos.

Finalmente el capítulo \textbf{Conclusión y trabajo futuro} (ver
\textbf{?@sec-conclusion}) se destina a recapitular los hallazgos
encontrados aventurando posibles explicaciones a los mismos y propone
líneas de investigación futuras en base a los resultados obtenidos.

\hypertarget{convenciones-usadas}{%
\subsection{Convenciones usadas}\label{convenciones-usadas}}

En este trabajo se ha evitado en la medida de lo posible el uso de
anglicismos traduciendo al español los términos ingleses cuando su uso
sea habitual en la publicación científica en español. No obstante,
algunos términos se han mantenido en inglés por no tener una traducción
fácil o habitual. Es el caso, por ejemplo, de \emph{odds} y de
\emph{odds ratio}. Se ha considerado que la utilización de los términos
en español, \emph{cuota} y \emph{razón de momios} respectivamente,
dificultan la comprensión y se ha preferido el inglés en estos casos y
otros similares.

Los nombres de los modelos estadísticos se han escrito con la inicial de
cada palabra en mayúscula. En los acrónimos generalmente se ha mantenido
su correspondencia en inglés. Por ejemplo, Modelo Lineal Generalizado
(GLM, Generalized Linear Model).

Para los nombres de las variables utilizadas en el modelado estadístico
se ha preferido el inglés. Por ejemplo, \emph{Treat} para referirse a
los subtitulados y \emph{Response} para las respuestas a los ítems de
las escalas de Likert. La justificación de esta decisión es evitar la
mezcla de idiomas en los resúmenes de los modelos o en los ejemplos de
código.

Este trabajo es reproducible. Para su elaboración se ha utilizado la
herramienta de publicación científica Quarto y el lenguaje de
programación R. Todas las figuras mostradas han sido generadas con R.



\end{document}
